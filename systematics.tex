\chapter{Systematic Uncertainties}\label{cha:systematics}

Systematic uncertainties arise from various sources in the analysis of $\Httllfull$.
They affect the event yields in the final distributions, which are used in the fit.
There are two types of systematic uncertainties: the \emph{shape uncertainties}, which have an impact on
the distributions of the used observables, and the \emph{normalization systematics}, which affect the
expected signal and background yields.

Systematic uncertainties can by classified by the type of their source.
Experimental systematics arise due to measurements of for example the luminosity or
efficiencies of object identification and reconstruction.
The calibration of the detectors, especially the calorimeters, is another contribution.
Another source are uncertainties related to theory predictions.
These depend for example on the choice of the QCD scale or PDF uncertainties.
Additionally, the data-driven estimation of background processes result in another class of systematic uncertainties.

For most of the systematic uncertainties the source is varied upwards or downwards within one standard deviation and the full
analysis is repeated on the modified input, in order to propagate the systematic uncertainty correctly.
Then, the impact on the shape of the observable used in the fit and the total signal and background yield is compared
with respect to the nominal case.
Systematic variations which have only a very minor impact are discarded by a \emph{pruning} procedure.

This chapter gives an overview of the systematic uncertainties which are considered in this analysis.
Their incorporation in the fit is discussed in \cref{sec:fit:nps}.

\section{Experimental uncertainties}\label{sec:systematics:exp}

\begin{description}[leftmargin=0cm]
    \item[Luminosity:] The uncertainty of the integrated luminosity is $\SI{\pm 2.1}{\percent}$ for the combined 2015 and 2016 dataset~\cite{LumiUncertRun2}.
        This value was derived in a procedure similar to the method described in~\cite{LumiUncertRun1}
        from a calibration of the luminosity scale using x-y beam-separation scans performed in August 2015 and May 2016.
    \item[Electrons:] The uncertainties on the electron trigger, identification, and isolation efficiencies
        are derived in $J/\Psi \to ee$ and $Z \to ee$ events. The relative variations are within \SI{0.5}{\percent}
        and \SI{5}{\percent}~\cite{ATLAS-CONF-2016-024,ElectronSFUncert}.
        Additionally, systematic variations on the electron energy resolution are applied, which is most of the
        time less than \SI{1}{\percent}~\cite{PERF-2013-05}.
    \item[Muons:] The muon momentum scale and energy resolutions are varied by $\pm 1 \sigma$
        in the event reconstruction. The variations are derived in $J/\Psi \to \mu\mu$ and $Z \to \mu\mu$
        are range between \SI{1.7}{\percent} and \SI{2.9}{\percent}.
        The systematic variations on the trigger, identification, and isolation efficiencies for muons
        are derived in the same events.
        The variations are between \SI{1}{\percent} and \SI{7}{\percent}~\cite{PERF-2015-10}.
    \item[Jet energy:] The jet uncertainties depend on the transverse momentum and $\eta$ of the jet.
        The uncertainties on the jet energy resolution (JER) and jet energy scale (JES) are obtained
        by smearing the nominal jet energy resolution.
        The JES uncertainty is \SI{6}{\percent} for jets with $\pt = \SI{20}{\GeV}$,
        decreases to \SI{1}{\percent} for jets between with a transverse momentum between \SI{200}{\GeV} and \SI{1800}{\GeV} and rises again to \SI{3}{\percent}
        for jets with higher $\pt$.
        For the JER and JES uncertainties the 11 and 19 parameter scheme is used, respectively.~\cite{ATL-PHYS-PUB-2015-015,PERF-2016-04}
    \item[Jet vertex tagger:] The jet vertex tagger (JVT) algorithm is used to suppress jets from pile-up events
        by applying a threshold on the output of the JVT algorithm, as described in \cref{sec:object_selection:jets}.
        The variation of this threshold is used as a systematic uncertainty.
    \item[b-tagging:] Scale factors are used to correct the $b$-tagging efficiencies and mistag rate for light flavour jets.
        This class of systematic variations refers to the variation of these scale factors.
        The systematic variation affects mainly the single-top and $t\overline{t}$ background.
    \item[Transverse missing energy:] For the systematic uncertainty on the missing transverse energy the soft-term
        contribution is smeared to match the data~\cite{METUncertainty}.
    \item[Pile-up reweighting:] In the generation of simulated events a generalized profile for the distribution of the number of interactions
        per bunch crossing (pile-up distribution) is used. To match with the observed profile of the 2015 and 2016 dataset, a correction factor
        of $1/1.16$ needs to be applied to the pile-up distribution..
        It was determined that the uncertainty for a $1\sigma$ variation is $1/(1.16 \pm 0.07)$.
        However, the more conservative estimation of $1/(1.16^{+0.07}_{-0.16})$ is used in this analysis.
\end{description}

\section{Uncerainties on data-driven background estimations}\label{sec:systematics:bkg}

The background of events with misidentified leptons (fake leptons) is estimated in this analysis in a data-driven way,
as described in \cref{sec:background_estimation:fakes}.
The systematic variations are constructed from variations of the efficiencies of the triggers, which are used to select the events.
Additionally, non-closure systematics are derived by comparing data events from the opposite sign and same sign fake control region.

\section{Theory uncerainties}\label{sec:systematics:theo}

\begin{description}[leftmargin=0cm]
    \item[Signal theory systmatics:] For the gluon--gluon fusion, vector boson fusion, and Higgs-Strahlung production
        modes of the Higgs boson several systematic variations are applied.
        For the uncertainties on the parton distribution function (PDF) several parametrizations of different PDF sets are compared~\cite{YR4}.
        Additionally, the value of the strong coupling constant $\alpha_s$ is varied, $\alpha_s(M_Z) = 0.118 \pm 0.0015$~\cite{YR4}.
        Discrepancies due to missing higher-order calculations of the cross-sections are estimated by the QCD scale uncertainty.
        Here, the factorization and renormalization scales are varied by a factor $2$ in both up and down direction.
        All these uncertainties are shape uncertainties.
    \item[$\mathbf{Z\tau\tau}$ theory systmatics:] For the $\Ztautau$ background there are variations applied on the PDF set and
        renormalization, factorization, CKK, and QSF scale.
    \item[$\mathbf{\Htt}$ branching ratio:] The uncertainty on the $\Htt$ branching ratio is $\pm \SI{5.7}{\percent}$~\cite{YR3}.
\end{description}

