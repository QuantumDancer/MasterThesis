\begin{abstractpage}
    \begin{abstract}{english}
        The Higgs boson was first observed in July 2012 by the ATLAS and CMS experiments based at CERN~\cite{HiggsDiscoveryATLAS,HiggsDiscoveryCMS}.
        In a combined measurement of the two collaborations the mass was determined to $m_H = (125.09 \pm 0.21 \text{(stat.)} \pm 0.11 \text{(sys.)})\,\text{GeV}$~\cite{MassCombinedMeas}.
        Further measurements confirmed the consistency with Standard-Model predictions of the Higgs boson.
        The $\Htt$ decay channel of the Higgs boson is the most sensitive decay channel to probe the Yukawa couplings of the Higgs boson.
        This makes it an interesting channel to analyze during the second data-taking period of the LHC starting in 2015.
        In this thesis a multivariate approach based on boosted decision trees is developed to increase the sensitivity
        with respect to the cut-based analysis of the $\Httllfull$ decay channel for a combined 2015 and 2016 dataset
        corresponding to an integrated luminosity of $\lumi = \SI{36.1}{\invfb}$ recorded with the ATLAS detector
        in proton--proton collisions at a center-of-mass energy of $\sqrt{s} = \SI{13}{\TeV}$.
        The BDT hyperparameters and observables used as input variables are optimized in a $k$-fold cross-validation approach.
        The expected sensitive of the cut-based analysis is $0.83\sigma$.
        Using the multivariate approach it was possible to increase the sensitive to $1.35\sigma$.
        Additionally, the expected uncertainties on the measurement of the signal strength $\mu$
        are reduced from $\pm 1.27$ in the cut-based analysis to $1 \pm 0.68$ in the multivariate analysis.
    \end{abstract}

    \begin{abstract}{ngerman}
    \begin{otherlanguage}{ngerman}
        Das Higgs-Boson wurde erstmals im Juli 2012 durch das ATLAS und CMS Experiment am CERN beobachtet~\cite{HiggsDiscoveryATLAS,HiggsDiscoveryCMS}.
        Das kombinierte Ergebnis der ATLAS und CMS Kollaborationen für die Masse des Higgs-Bosons ist $m_H = (125.09 \pm 0.21 \text{(stat.)} \pm 0.11 \text{(sys.)})\,\text{GeV}$~\cite{MassCombinedMeas}.
        Weitere Messungen konnten die Übereinstimmung mit dem durch das Standard Model vorhergesagten Higgs-Bosons bestätigen.
        Der sensitivste Zerfallskanal für die Messung der Yukawa-Kopplung des Higgs-Bosons ist der $\Htt$ Zerfallskanal.
        Deshalb ist die Analyse dieses Zerfallskanals eine wichtige Aufgabe des zweite Datenerfassungszeitraumes am LHC, welcher im Jahre 2015 gestartet wurde.
        In dieser Masterarbeit wird eine multivariate Methode basierend auf geboosteten Entscheidungsbäumen (BDTs) entwickelt, welche
        die Empfindlichkeit auf das Signal im Vergleich zur Schnitt-basierten Analyse des $\Httllfull$ Prozesses
        in einem kombinierten Datensatz aus den Jahren 2015 und 2016 mit einer integrierten Luminosität von $\lumi = \SI{36.1}{\invfb}$,
        welcher mit dem ATLAS Detektor bei einer Schwerpunktsenergie von $\sqrt{s} = \SI{13}{\TeV}$ in Proton--Proton Kollisionen aufgenommen wurde, verbessert.
        Die Hyperparameter und Observablen, die für die BDTs benutzt werden, werden mit einer $k$-fachen Kreuzvalidierung optimiert.
        Die erwartete Empfindlichkeit der Schnitt-basierten Analyse von $0.83\sigma$ konnte mit dem multivariaten Verfahren
        auf $1.35\sigma$ erhöht werden.
        Des Weiteren wird mit der multivariaten Methode die erwartete Unsicherheit auf die Messung der Signalstärke von
        $\pm1.27$ auf $\pm0.68$ reduziert.
    \end{otherlanguage}
    \end{abstract}
\end{abstractpage}
