% !TEX program = pdflatex
% !TEX encoding = UTF-8
\documentclass[12pt,border=12pt]{standalone}

% language, encoding, font
\usepackage[T1]{fontenc} % Use 8-bit encoding that has 256 glyphs
\usepackage[utf8]{inputenc}
\usepackage{lmodern}

% math
\usepackage{amsmath}
\usepackage{amssymb}
\usepackage{amsfonts}
\usepackage{mathtools}
\usepackage{tensor}
\usepackage{commath}
\usepackage{bm} % bold in math mode
\usepackage{units}
\usepackage{color}
\usepackage{feynmp-auto}

\begin{document}

\begin{fmffile}{fmf_qcd_1}
    \begin{fmfgraph*}(100,75)
        \fmfstraight
        \fmftop{i1,v1,o1}
        \fmfbottom{i2,v2,o2}
        \fmf{fermion}{i2,v2,v1,i1}
        \fmf{gluon}{v1,o1}
        \fmf{gluon}{v2,o2}

        \fmfdot{v1,v2}
        \fmfv{label=$\overline{q}$, label.angle=180}{i1}
        \fmfv{label=$q$, label.angle=180}{i2}
        \fmfv{label=$g$, label.angle=0}{o1}
        \fmfv{label=$q$, label.angle=0}{o2}
    \end{fmfgraph*}
\end{fmffile}

\end{document}

% vim: filetype=tex
