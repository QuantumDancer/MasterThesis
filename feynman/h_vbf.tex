% !TEX program = pdflatex
% !TEX encoding = UTF-8
\documentclass[12pt,border=12pt]{standalone}

% language, encoding, font
\usepackage[T1]{fontenc} % Use 8-bit encoding that has 256 glyphs
\usepackage[utf8]{inputenc}
\usepackage{lmodern}

% math
\usepackage{amsmath}
\usepackage{amssymb}
\usepackage{amsfonts}
\usepackage{mathtools}
\usepackage{tensor}
\usepackage{commath}
\usepackage{bm} % bold in math mode
\usepackage{units}
\usepackage{color}
\usepackage{feynmp-auto}

\begin{document}

\begin{fmffile}{fmf_h_vbf}
    \begin{fmfgraph*}(90,75)
        \fmftop{i1,d1,o1}
        \fmfright{o2}
        \fmfbottom{i2,d2,o3}
        \fmf{fermion}{i1,v1}
        \fmf{fermion}{i2,v2}
        \fmf{fermion}{v1,o1}
        \fmf{fermion}{v2,o3}
        \fmf{phantom, label=$W,,Z$, label.dist=-5, tension=0.2}{v1,v2}
        \fmffreeze
        \fmf{boson}{v1,v3,v2}
        \fmf{dashes}{v3,v4}
        \fmf{phantom, tension=5}{v4,o2}
        \fmfdot{v1,v2}
        \fmfv{label=$H$, label.angle=0}{v4}
        \fmfv{label=$q$, label.angle=90}{v1}
        \fmfv{label=$q$, label.angle=-90}{v2}
    \end{fmfgraph*}
\end{fmffile}

\end{document}

% vim: filetype=tex
