\chapter{Boosted Decision Trees}\label{cha:boosted_decision_trees}

In most \hep{} analyses the goal is to measure a rare process (signal), which is contained in a
large dataset of measurements of particle collisions.
However, most of the measured collisions (events) originate from other processes (background).
One goal of an analysis is to separate the signal events from the background events in order to obtain a pure
measurement of the desired process.

An established approach to separate signal and background is to restrict some physical quantities of the collisions
to a given range.
Such a restriction is called cut.
Usually different quantities are cut on to limit different sources of background.
Such an approach is described in \cref{cha:event_selection}.

Another way to achieve separation between signal and background is with the help of \ml{} algorithms.
The most common \ml{} algorithms in \hep{} are \bdts{} and \nns{}.

This chapter will describe what \bdts{} are and how they are built.
First, \glspl{DT} are introduced in \cref{sec:bdt:decision_trees}, since they are the foundation for \bdts.
Then, the concept of boosting is explained in \cref{sec:bdt:boosting} and different boosting algorithms are presented.
Finally, in \cref{sec:bdt:hyperparameters} a summary of all possible configuration parameters of \bdts{} is given.

The contents of this chapter are mainly based on Refs.~\cite{ML_HASTIE,ML_TMVA}.

\section{Decision Trees}\label{sec:bdt:decision_trees}

A \gls{DT} has a binary tree-like structure as shown in \cref{fig:decision_tree}.
The root node contains all signal and background events.
Each internal node represents

\begin{figure}[htb]
    \begin{center}
        % \includegraphics[width=0.8\linewidth]{name.ext}
        \caption{TODO}
        \label{fig:decision_tree}
    \end{center}
\end{figure}

% TODO comparison with cuts

\section{Boosting}\label{sec:bdt:boosting}

\subsection{AdaBoost}\label{sub:bdt:boosting:adaboost}

\subsection{A General Approach to Boosting}\label{sub:bdt:boosting:general}

\subsection{Gradient Boost}\label{sub:bdt:boosting:gradient_boost}

\section{Hyperparameters}\label{sec:bdt:hyperparameters}

nCuts
