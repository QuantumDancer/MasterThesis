\chapter{Introduction}\label{cha:introduction}

In elementary particle physics the fundamental constituents of nature and their interactions are investigated.
In the 1960s and 1970s the Standard Model (SM) of particle physics was developed to provide an accurate description of all elementary particles which are known today
and their fundamental interactions.
The particles can be classified based on their spin: \emph{fermions} have a half-integer spin and \emph{bosons} have an integer spin.
Three out of the four fundamental interactions, the electromagnetic, strong, and weak interaction, are incorporated in the SM as
relativistic quantum field theories.
No formulation of gravity as a relativistic quantum field theory is found yet.
However, the influence of gravity at the energy scales considered in particle physics is minor and therefore can be neglected.

With the help of the Standard Model it was possible to predict several particles, which were later found in nature,
for example the massive $W^\pm$- and $Z^0$-bosons which were first observed in 1983~\cite{ZDiscovery,WDiscovery,ZeeDiscovery,WeDiscovery}.
In earlier models, which attempted to describe elementary particles in a quantum field theory,
no mass terms for fermions and bosons were included, which stands in contrast to measurements of fermion and boson masses.
For example, the $W^\pm$- and $Z^0$-bosons have a mass of $m_{W^\pm} = \SI{80.4}{\GeV}$ and $m_{Z^0} = \SI{91.2}{\GeV}$, respectively~\cite{PDG}.
This conflict was solved by the introduction of the
Englert--Brout--Higgs--Guralnik--Hagen--Kibble mechanism~\cite{HiggsMecha1,HiggsMecha2,HiggsMecha3,HiggsMecha4,HiggsMecha5,HiggsMecha6} (short: Higgs mechanism)
which uses the principle of spontaneous symmetry breaking.
A new scalar field, the Higgs field, is introduced.
The interaction of fermions and bosons with the vacuum expectation value of the Higgs field leads to the required mass terms of the particles.
The particle associated with the Higgs field is the \emph{Higgs boson}.
During the last decades great effort was put into the search for the Higgs boson at different collider experiments, until it
was found with the ATLAS\footnote{\textbf{A} \textbf{T}oroidal \textbf{L}HC \textbf{A}pparatu\textbf{S}} and
CMS\footnote{\textbf{C}ompact \textbf{M}uon \textbf{S}olenoid} experiments at
CERN\footnote{\textbf{C}onseil \textbf{E}uropéen pour la \textbf{R}echerche \textbf{N}ucléaire} in 2012~\cite{HiggsDiscoveryATLAS,HiggsDiscoveryCMS}.
The mass of the newly found particle was determined to be $m_H = 125.09 \pm 0.21 \text{(stat.)} \pm 0.11 \text{(sys.)}\,\text{GeV}$
in a combined measurement of the ATLAS and CMS collaboration~\cite{MassCombinedMeas}.
After the Higgs boson was found at CERN, Peter Higgs and Francois Englert were awarded the Nobel Prize in 2013 for the formulation of
the underlying theory.

The $H \to \tau\tau$ decay mode is an important decay channel of the Higgs boson, since it it provides the most sensitive measurement
of the Yukawa couplings of the Higgs boson.
Additional, the $H \to \tau\tau$ decay channel can be used to probe a potential CP mixing and lepton-flavour violating decays of the Higgs boson.
During the first data-taking period at the LHC in 2011 and 2012 evidence for this decay channel was found at the ATLAS experiment
at the level of $4.5\sigma$~\cite{HTauTauRun1}.
The signal strength $\mu = \sigma_\text{obs} / \sigma_\text{SM}$ was determined to be $\mu = 1.43\errud{0.43}{0.37}$.
Due to limited statistic the observation where a sensitivity of $5\sigma$ is required could not be made.

In 2015 the second data taking period started at the LHC with an increased center-of-mass energy of $\sqrt{s} = \SI{13}{\TeV}$ and integrated luminosity.
This allows for a more precise measurement of the signal strength of the $\Htt$ channel and measurements in the individual production modes.
However, due to the new conditions the analysis strategy needs to be reoptimized.
In this thesis the $\Httllfull$ decay channel is considered, where both leptons decay leptonically, with the
combined 2015 and 2016 dataset of the second data-taking period corresponding to an integrated luminosity of $\lumi = \SI{36.1}{\invfb}$
at a center-of-mass energy of $\sqrt{s} = \SI{13}{\TeV}$.
A method is developed to increase the sensitivity in this decay channel with the help of multivariate techniques.
More precisely, the usage of boosted decision trees (BDTs), a machine-learning algorithm, is investigated.

This thesis is structured as follows.
First the Standard Model is introduced and an overview of the current status of Higgs-boson property measurements is given in \cref{cha:theory}.
In \cref{cha:setup} the LHC and ATLAS experiment are described.
The signal and background processes considered in the analysis of $\Httllfull$ are discussed in \cref{cha:processes}, followed by an overview
of the reconstruction of physical objects in \cref{cha:object_selection}.
The event selection is described in \cref{cha:event_selection} and the strategy to estimate specific backgrounds
is presented in \cref{cha:background_estimation}.
An overview of the theoretical aspects of BDTs is given in \cref{cha:bdt} and their application to the analysis of the $\Httllfull$ process
is discussed in \cref{cha:mva}.
In \cref{cha:systematics} the systematic uncertainties are presented.
The thesis closes with a description of the statistical procedure and a discussion of the results in \cref{cha:fit}, followed
by a summary of the analysis and an outlook for further studies in \cref{cha:conclusion}.
