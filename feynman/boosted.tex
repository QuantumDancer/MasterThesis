% !TEX program = pdflatex
% !TEX encoding = UTF-8
\documentclass[12pt,border=12pt]{standalone}

% language, encoding, font
\usepackage[T1]{fontenc} % Use 8-bit encoding that has 256 glyphs
\usepackage[utf8]{inputenc}

% math
\usepackage{amsmath}
\usepackage{amssymb}
\usepackage{amsfonts}
\usepackage{mathtools}
\usepackage{tensor}
\usepackage{commath}
\usepackage{bm} % bold in math mode
\usepackage{units}
\usepackage{color}
\usepackage{feynmp-auto}

\begin{document}

\begin{fmffile}{fmf_boost}
    \begin{fmfgraph*}(130,75)
        \fmftop{i1,d1,o1}
        \fmfright{o2}
        \fmfbottom{i2,d2,o3}
        \fmf{phantom}{i1,v1,d3,o1}
        \fmf{phantom}{i2,v2,d4,o3}
        \fmffreeze
        \fmf{gluon}{i1,v1}
        \fmf{gluon}{i2,v2}
        \fmf{fermion, label=$t,,b$, label.side=left}{v1,v2}
        \fmf{fermion}{v2,v3,v1}
        \fmf{dashes, tension=2}{v3,v4}
        \fmf{phantom, tension=5}{v4,o2}
        \fmffreeze
        \fmfright{r1,r2}
        \fmfforce{(0.52w,0.633h)}{r1}
        \fmfforce{(0.75w,0.95h)}{r2}
        \fmf{gluon}{r1,r2}
        \fmfdot{v1,v2,v3,r1}
        \fmfv{label=$g$, label.angle=180}{i1}
        \fmfv{label=$g$, label.angle=180}{i2}
        \fmfv{label=$g$, label.angle=45}{r2}
        \fmfv{label=$H$, label.angle=0, label.dist=1}{v4}
    \end{fmfgraph*}
\end{fmffile}

\end{document}

% vim: filetype=tex
