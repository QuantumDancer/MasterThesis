\chapter{Signal and Background Processes}\label{cha:processes}

This chapter discusses shortly the signal process which is considered in this analysis and gives an overview
over the relevant background contributions.
For a precise measurements it is important that the background processes are understood and modeled well.
In \cref{sec:processes:mc} the used software to predict and simulate signal and background processes is discussed.

\section{Signal Process}\label{sec:processes:signal}

The Higgs boson has several production modes and decay channels, as discussed in \cref{sec:theory:higgs}.
In this analysis the four main production mechanisms, namely the gluon--gluon fusion, vector-boson fusion, Higgs-Strahlung, and
top-quark pair associated production are considered, which are explained in more detail in \cref{sub:theory:higgs:production}.
The two dominant production modes are gluon--gluon fusion and vector-boson fusion.

The decay of the Higgs boson into two $\tau$-leptons, $\Htt$ is analyzed, which has a branching ratio of
$\SI{6.272}{\percent}$ for a mass of the Higgs boson of $m_H = \SI{125}{\GeV}$.
Since $\tau$-leptons have a very short mean decay lifetime of $\SI{2.9e-13}{\s}$ they cannot be detected
directly.
It is only possible to detect $\tau$-leptons by reconstructing their decay products.
A $\tau$-lepton can decay either into leptons (electrons or muons) or hadrons (combinations of charged and neutral pions)
via electroweak interactions, with a branching ratio of \SI{35}{\percent} and \SI{65}{\percent}, respectively~\cite{PDG}.
Thus, the $\Htt$ decay can be categorized into three subchannels, depending on the final state of the decaying $\tau$-leptons.
The focus of this analysis lies on the full-leptonic decay channel, $\Httllfull$, where both $\tau$-leptons decay leptonically.
This decay channel has a branching ratio of \SI{12}{\percent}.
The other two decay channels are the semi-leptonic and full-hadronic channel with one lepton and one hadron or two hadrons
in the final state, respectively.
The corresponding branching ratios are \SI{46}{\percent} and \SI{42}{\percent}.

\section{Background Processes}\label{sec:processes:background}

At the LHC a lot of different interactions can happen during the collisions.
Some processes have a similar detector signature as the signal process, the so-called \emph{background}.
There are two kinds of background processes, \emph{reducible} and \emph{irreducible} ones.
Irreducible backgrounds have the same final state as the signal process (i.e.\ same number of leptons, jets, $b$-jets, etc.), which makes the separation very hard.
For an accurate analysis, those backgrounds need to be modelled by simulations or estimated in a data driven way.
But also processes with a different final state topology can contribute to the background.
Errors in the identification of physical objects or reconstruction of the missing transverse energy can lead
to a signal-like detector signature, even if the real event topology is different from the signal process.
These processes are called reducible.
In the following sections all background processes are discussed, which are relevant to this analysis.

\subsection{$Z$ Boson Production in Association with Jets}\label{sub:processes:z}

One of the most important background processes is the production of a $Z$-boson or
virtual photon, with a subsequent decay into $\tau$-leptons or light leptons (electrons or muons).
This background features both an irreducible and a reducible part.
The $Z / \gamma^* \to \tau\tau \to \ell\ell + 4 \nu$ background is irreducible, due to the same final state topology.
If the $Z$-boson or virtual photon decay directly into electrons or muons, no transverse energy is produced.
However, additional jets in the final state can lead to a misreconstruction of the missing transverse energy.
Example Feynman diagrams of the production of a $Z$-boson with up to $2$ additional jets in the final state are
shown in \cref{fig:processes:z}.
\todo{Explain EW, include Feynman diagram}

\begin{figure}[htb]
    \centering
    \includegraphics[width=0.3\textwidth]{./feynman/z_0jet.pdf}
    \includegraphics[width=0.3\textwidth]{./feynman/z_1jet.pdf}
    \includegraphics[width=0.3\textwidth]{./feynman/z_2jet.pdf}
    \caption{Example Feynman diagrams for $Z$-boson production with up to two associated jets.}\label{fig:processes:z}
\end{figure}

\subsection{Diboson production}\label{sub:processes:diboson}

The production of $WW$-, $WZ$-, and $ZZ$-diboson pairs is combined in the
diboson background.
Here both $W$- and $Z$-bosons can decay either leptonically or hadronically.
The most important contribution comes from $WW$-boson decays, $WW \to 2 \ell 2 \nu$, since they have
the same final state as the signal process.
\cref{fig:processes:diboson} shows Feynman diagrams for the different diboson production mechanisms.
Additionally, the decay of Higgs bosons into a pair of $W$-bosons is also considered as background.

\begin{figure}[htb]
    \centering
    \includegraphics[width=0.3\textwidth]{./feynman/diboson_ww.pdf}
    \includegraphics[width=0.3\textwidth]{./feynman/diboson_wz.pdf}
    \includegraphics[width=0.3\textwidth]{./feynman/diboson_zz.pdf}
    \caption{Example Feynman diagrams for dominant diboson production modes.}\label{fig:processes:diboson}
\end{figure}


\subsection{Single Top-Quark and Top-Quark Pair Production}\label{sub:processes:top}

Another important background is the production of one or two top-quarks, whose decay is accompanied by
large amounts of jets.
Single top-quarks can be produced both in the $s$- and $t$-channel and in association with a $W$-boson, as shown in \cref{fig:processes:stop}.
The top-quarks decay in almost all cases into a $b$-quark and $W$-boson.
Since the $b$-quark can also decay into a $W$-boson and a lighter quark, two leptons and missing transverse energy can be in the final state due to
the decay of the $W$-bosons.
Additionally, decays of $B$-hadrons can also provide a prompt lepton in the final state.

\begin{figure}[htb]
    \centering
    \includegraphics[width=0.3\textwidth]{./feynman/stop_1.pdf}
    \includegraphics[width=0.3\textwidth]{./feynman/stop_2.pdf}
    \includegraphics[width=0.3\textwidth]{./feynman/stop_3.pdf}
    \caption{Example Feynman diagrams for single top-quark production in the $s$-channel (left),
             $t$-channel (middle), and $tW^\pm$ production (right).}\label{fig:processes:stop}
\end{figure}

The production of top-quark pairs is however the more dominant part of this background.
Top-quark pairs can be produced in processes with quarks and gluons in the initial state, which can be seen in \cref{fig:processes:ttbar}.
At the LHC the $gg \to t\overline{t}$ processed dominate, due to the high values of the gluon PDF in protons at low values of the
momentum fraction $x$.\todo{Numbers, Ref.}\
The decay chain form top-quarks to leptons is described above.

\begin{figure}[htb]
    \centering
    \includegraphics[width=0.3\textwidth]{./feynman/ttbar_1.pdf}
    \includegraphics[width=0.3\textwidth]{./feynman/ttbar_2.pdf}
    \includegraphics[width=0.3\textwidth]{./feynman/ttbar_3.pdf}
    \caption{Example Feynman diagrams for the production of top-quark pairs, $t\overline{t}$.}\label{fig:processes:ttbar}
\end{figure}

\subsection{QCD Multi-Jet Production}\label{sub:processes:qcd}

Because protons are collided at the LHC, QCD interactions with outgoing quarks and gluons have a high cross-section.
The quarks and gluons create jets due to hadronization, which sometimes are misidentified as leptons.
With the additional misreconstruction of missing transverse energy some events have a signal-like event topology.
Example Feynman diagrams for QCD multi-jet processes are shown in \cref{fig:processes:qcd}.

\begin{figure}[htb]
    \centering
    \includegraphics[width=0.3\textwidth]{./feynman/qcd_1.pdf}
    \includegraphics[width=0.3\textwidth]{./feynman/qcd_2.pdf}
    \includegraphics[width=0.3\textwidth]{./feynman/qcd_3.pdf}
    \caption{Example Feynman diagrams for QCD multi-jet production.}\label{fig:processes:qcd}
\end{figure}

\section{Monte Carlo Simulations}\label{sec:processes:mc}

\todo{Some references are missing}
All signal processes except for the production mode associated with a top-quark pair are modelled with
\textsc{Powheg-Box v2}~\cite{PowhegBox2} interfaced to \textsc{Pythia8}~\cite{Pythia8}.
For gluon--gluon fusion and vector-boson fusion the \textsc{NNLOPS}~\cite{NNLOPS} PDF set is used for the matrix element and the \textsc{AZNLO CTEQ6L1}~\cite{CTEQ6} PDF tune for the
modelling of non-perturbative effects, while for the Higgs-boson production associated with a vector boson the
\textsc{NNPDF3.0} and \textsc{AZNLO} tune is used.
For the simulation of $t\overline{t}H$ events \textsc{aMC@NLO} combined with \textsc{Pythia8} is used.
Here the PDFs are described by \textsc{NNPDF3.0}~\cite{NNPDF30}.
All decay channels of the $\Htt$ decay are included in the signal samples.

Events originating from the $Z/\gamma^*$ and diboson background are generated by
\textsc{Sherpa 2.2.1}~\cite{Sherpa,Gleisberg:2008fv,Cascioli:2011va,Schumann:2007mg,Hoeche:2012yf}
with the \textsc{NNPDF30NNLO}~\cite{NNPDF30} PDF tune.
The electroweak contributions of $Z/\gamma^*$ are calculated seperately with the same settings.
To simulate events of the top-quark background a combination of \textsc{Powheg} and \textsc{Pythia6}~\cite{Pythia6}
is used with the \textsc{CT10} PDF set and \textsc{Perugia 2012} tune.
The $H\to W^+W^-$ process is generated by \textsc{Powheg} and \textsc{Pythia?} with the \textsc{CT10} PDF set.
The generators and cross-sections for all processes are listed in \cref{tab:processes:mc}.

For all events the full response of the ATLAS detector is simulated~\cite{SOFT-2010-01} with the help of \textsc{Geant4}~\cite{Geant4}.
Pile-up events are generated with \textsc{Pythia8} and overlaid corresponding to the pile-up profile.

% \pagebreak[4]
% \global\pdfpageattr\expandafter{\the\pdfpageattr/Rotate 90}

\todo{Table needs to be completed}
\begin{sidewaysfigure}
    \centering
    \caption{Signal and background processes used in the $\Httll$ analysis. The product of cross-section and branching ratio corresponds to a $2015+2016$ dataset
    at $\sqrt{s} = \SI{13}{\TeV}$. The generators and PDF sets which are used to predict the events are also listed.}\label{tab:processes:mc}
    \begin{tabular}{lllll}
        \toprule
        Signal  & $\sigma \times \mathcal{B}$ [\si{\pico\barn}] & Order & Generator & PDF tune \\ 
        $m_H = \SI{125}{\GeV}$ & at $\sqrt{s} = \SI{13}{\TeV}$ & & & \\ \midrule
        ggF $\Htt$ & 3.0469   & & Powheg + Pythia8 & NNLOPS + AZNLO CTEQ6L1 \\
        VBF $\Htt$ & 0.23721  & & Powheg + Pythia8 & NNLOPS + AZNLO CTEQ6L1 \\
        WH  $\Htt$ & 0.086102 & & Powheg + Pythia8 & NNPDF30 + AZNLO \\
        ZH  $\Htt$ & 0.055438 & & Powheg + Pythia8 & NNPDF30 + AZNLO \\
        ttH $\Htt$ & 0.5071   & & aMC@NLO + Pythia8 & NNPDF3.0 \\ \midrule
        Background & & & & \\ \midrule
        $Z/\gamma^* \to e^+e^-$          & & & Sherpa 2.2.1 & NNPDF30NNLO \\
        $Z/\gamma^* \to e^+e^-$ EW       & & & Sherpa 2.2.1 & NNPDF30NNLO \\
        $Z/\gamma^* \to \mu^+\mu^-$      & & & Sherpa 2.2.1 & NNPDF30NNLO \\
        $Z/\gamma^* \to \mu^+\mu^-$ EW   & & & Sherpa 2.2.1 & NNPDF30NNLO \\
        $Z/\gamma^* \to \tau^+\tau^-$    & & & Sherpa 2.2.1 & NNPDF30NNLO \\
        $Z/\gamma^* \to \tau^+\tau^-$ EW & & & Sherpa 2.2.1 & NNPDF30NNLO \\
        Diboson & & & Sherpa 2.2.1 & NNPDF30NNLO \\
        single $t$, $s$-channel & & & Powheg + Pythia6 & CT10 + Perugia 2012 \\
        single $t$, $t$-channel & & & Powheg + Pythia6 & CT10 + Perugia 2012 \\
        $tW$ & & & Powheg + Pythia6 & CT10 + Perugia 2012 \\
        $t\overline{t}$  & & & Powheg + Pythia6 & CT10 + Perugia 2012 \\
        ggF $H \to W^+ W^-$ & & & Powheg + Pythia? & CT10 \\
        VBF $H \to W^+ W^-$ & & & Powheg + Pythia? & CT10 \\
        \bottomrule
    \end{tabular}
\end{sidewaysfigure}
